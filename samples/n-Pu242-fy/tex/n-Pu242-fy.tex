\begin{samplecase}
{\bf Neutron multiplicities and fission yields for n + ${}^{242}$Pu}\newline
TALYS contains an implementation of several versions of the GEF code by Karl-Heinz Schmidt and Beatriz Jurado. Part of GEF has been translated by Vasily Simutkin and Michail Onegin into a Fortran subroutine.
In this sample case, the prompt neutron multiplicity as a function of mass, 
$\overline{\nu}(A)$, number of neutrons, $P({\nu})$, and the total average,
$\overline{\nu}$,
are calculated, as well as the pre- and post-neutron fission yields as a 
function of $N$ and $A$.
The following input file is used

\VerbatimInput{\samples n-Pu242-fy/org/talys.inp}

Note the comment given at the end of the input file. Note also that a hardwired energy grid is used through {\it n0-20.grid}. The prompt average neutron 
multiplicity $\overline{\nu}$ is given in the output file {\it nubar.tot}. 
Fig. \ref{nubar} shows a comparison with experimental data and some of the 
world nuclear data libraries.
Fig. \ref{nuA} presents the distribution of $\overline{\nu}(A)$ as a function of fission product mass, for an incident energy of 1 MeV. This is given in output 
file {\it nuA001.000.fis}.
Fig. \ref{pnu} presents the distribution of $\nu$ as a function of number of neutrons, for an incident energy of 1 MeV. This is given in output 
file {\it Pnu001.000.fis}.
\end{samplecase}
\begin{figure}
\centering\includegraphics[scale=0.5,angle=270]{nubar}
\caption{Average prompt neutron multiplicity $\overline{\nu}$ for n + $^{242}$Pu, with the GEF model, compared with experimental data and nuclear data libraries.}
\label{nubar}
\end{figure}
\begin{figure}
\centering\includegraphics[scale=0.5,angle=270]{nuA}
\caption{Average prompt neutron multiplicity $\overline{\nu}(A)$, as a function of fission product mass, for n + $^{242}$Pu, with the GEF model.}
\label{nuA}
\end{figure}
\begin{figure}
\centering\includegraphics[scale=0.5,angle=270]{pnu}
\caption{Prompt neutron multiplicity distribution $P(\nu )$, as a function of 
the number of neutrons, for n + $^{242}$Pu, with the GEF model.}
\label{pnu}
\end{figure}
