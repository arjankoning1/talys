\begin{samplecase}
{\bf All results for 14 MeV n + ${}^{93}$Nb}\newline
We have included 9 different versions of this sample case, in order to give an 
impression of the various types of information that can be retrieved from 
TALYS.  Most, but not all, output options will be described, while the 
remainder will appear in the other sample cases. We have chopped the sample 
output after column 80 to let it fit within this manual. We suggest to consult 
the output files in the {\em samples/} directory for the full results.
Note that in later versions of TALYS, more and more output is no longer given in the main large output file, 
but in separate output files with the results per reaction channel etc.
\subsubsection{Case a: The simplest input file}
The first sample problem concerns the simplest possible TALYS calculation.
Consider the following input file that produces the
results for a 14 MeV neutron on ${}^{93}$Nb:

\VerbatimInput{\samples n-Nb093-14MeV-general/org/talys.inp}

This input file called {\em input} can simply be run as follows:\newline

{\bf talys $<$ talys.inp $>$ talys.out}\newline

An output file of TALYS consists of several blocks. Whether these blocks are 
printed or not depends on the status of the various keywords that were 
discussed in Chapter \ref{input}. By 
default, the so-called main output is always given (through the default
{\bf outmain y}), and we discuss this output in the present sample case.
For a single incident energy, a default calculation gives the most important 
cross sections only. ``Most important'' is obviously subjective, and probably
every user has an own opinion on what should always appear by default in 
the output. We will demonstrate in the other sample problems how to extract all 
information from TALYS. 
Some of the information in the main output file in this sample case can only be obtained
with the keyword {\bf outall y}. Otherwise, it is in separate output files.
The output file starts with a display of the version of TALYS you are using,
the name of the authors, and the Copyright statement. Also the physics 
dimensions used in the output are given:

{\small \begin{verbatim}

    TALYS-2.2 (Version: December 29, 2025)

 Copyright (C) 2025  A.J. Koning, S. Hilaire and S. Goriely

 Dimensions - Cross sections: mb, Energies: MeV, Angles: degrees
\end{verbatim} } \renewcommand{\baselinestretch}{1.07}\small\normalsize
\noindent
The next output block begins with:

{\small \begin{verbatim}

########## USER INPUT ##########
\end{verbatim} } \renewcommand{\baselinestretch}{1.07}\small\normalsize
\noindent
Here, the first section of the output is a print of the keywords/input 
parameters. This is done in two steps: First, in the block

{\small \begin{verbatim}

USER INPUT FILE

#
# n-nb093-14mev
#
# general
#
projectile n
element nb
mass 93
energy 14.
\end{verbatim} } \renewcommand{\baselinestretch}{1.07}\small\normalsize
\noindent
an exact copy of the input file as given by the user is returned. Next, in
the block

{\small \begin{verbatim}

USER INPUT FILE + DEFAULTS

Keyword           Value   Variable     Explanation

#
# Four main keywords
#
projectile          n     ptype0       type of incident particle
element            Nb     Starget      symbol of target nucleus
mass               93     mass         mass number of target nucleus
energy             14.000 eninc        incident energy in MeV
#
# Basic physical and numerical parameters
#
ejectiles g n p d t h a   outtype      outgoing particles
.............
\end{verbatim} } \renewcommand{\baselinestretch}{1.07}\small\normalsize
\noindent
a table with all keywords is given, 
not only the ones that you have specified in the input file, but also all the 
defaults that are set automatically. The corresponding Fortran variables are 
also printed, together with a short explanation of their meaning. This table 
can be helpful as a guide to change further input parameters for a next run. 
You may also copy and paste the block directly into your next input file.

In the next output block

{\small \begin{verbatim}

########## BASIC REACTION PARAMETERS ##########
 
Projectile           : neutron     Mass in a.m.u.      :   1.008665
Target               :  93Nb       Mass in a.m.u.      :  92.906373
 
Included channels:
                    gamma
                    neutron
                    proton
                    deuteron
                    triton
                    helium-3
                    alpha
 
1 incident energy (LAB):
 
    14.000
 
Q-values for binary reactions:
 
Q(n,g):  7.22755
Q(n,n):  0.00000
Q(n,p):  0.69112
Q(n,d): -3.81879
Q(n,t): -6.19635
Q(n,h): -7.72313    
Q(n,a):  4.92559
\end{verbatim} } \renewcommand{\baselinestretch}{1.07}\small\normalsize
\noindent
we print the main parameters that characterize the nuclear 
reaction: the projectile, target and their masses and the outgoing particles 
that are included as competitive channels. The incident energy or range of 
incident energies in the LAB system is given together with the binary reaction
Q-values.

The block with final results starts with

{\small \begin{verbatim}

########## RESULTS FOR E=  14.00000 ##########

 Energy dependent input flags

 Width fluctuations (flagwidth)            : n
 Unresolved resonance parameters (flagurr) : n
 Preequilibrium (flagpreeq)                : y
 Multiple preequilibrium (flagmulpre)      : n
 Number of continuum excitation energy bins: 40
\end{verbatim} } \renewcommand{\baselinestretch}{1.07}\small\normalsize
\noindent
with no further information for the present sample case since no further output 
was requested.
When all nuclear model calculations are done, the most important cross sections 
are summarized in the main part of the output,
in which we have printed the center-of-mass energy, the main (total) cross 
sections.
the inclusive binary cross sections $\sigma^{inc,bin}_{n,k}$, see 
Eq.~(\ref{incbin}), the total particle production cross sections 
$\sigma_{n,xn}$ of Eq.~(\ref{cumu}) and the multiplicities $Y_{n}$ of
Eq.~(\ref{Yn}),
and the residual production cross sections. The latter are given first
per produced nuclide and isomer. Next, nuclides with the same mass are summed 
to give mass yield curves. Also, the sum over all the residual cross sections
is compared with the non-elastic cross section. Obviously, these two values
should be approximately equal.

{\small \begin{verbatim}

########### REACTION SUMMARY FOR E=  14.00000 ###########

Center-of-mass energy:   13.849

1. Total (binary) cross sections

Total           = 3.98195E+03
  Shape elastic   = 2.21132E+03
  Reaction        = 1.77063E+03
    Compound elastic= 6.02817E-04
    Non-elastic     = 1.77063E+03
      Direct          = 3.09942E+01
      Pre-equilibrium = 4.19247E+02
      Giant resonance = 5.69210E+01
      Compound non-el = 1.26347E+03
    Total elastic   = 2.21132E+03

2. Binary non-elastic cross sections (non-exclusive)

gamma   = 2.79389E+00
neutron = 1.69529E+03
proton  = 3.85805E+01
deuteron= 4.96449E+00
triton  = 1.81757E-01
helium-3= 7.29505E-10
alpha   = 2.88177E+01

3. Total particle production cross sections

gamma   = 2.21097E+03    Multiplicity= 1.24869E+00
neutron = 3.08736E+03    Multiplicity= 1.74365E+00
proton  = 4.10702E+01    Multiplicity= 2.31953E-02
deuteron= 4.96449E+00    Multiplicity= 2.80380E-03
triton  = 1.81757E-01    Multiplicity= 1.02651E-04
helium-3= 7.29505E-10    Multiplicity= 4.12003E-13
alpha   = 2.93826E+01    Multiplicity= 1.65944E-02

4. Residual production cross sections

  a. Per isotope

  Z   A  nuclide    total     level   isomeric    isomeric    lifetime
                cross section       cross section  ratio

  41  94 ( 94Nb) 1.19960E+00    0    5.71296E-01  0.47624
                                1    6.28304E-01  0.52376   3.75800E+02 sec. 
  41  93 ( 93Nb) 3.14864E+02    0    2.77770E+02  0.88219
                                1    3.70943E+01  0.11781   5.08700E+08 sec. 
  40  93 ( 93Zr) 2.97680E+01    0    2.97680E+01  1.00000
  41  92 ( 92Nb) 1.37917E+03    0    8.65640E+02  0.62765
                                1    5.13527E+02  0.37235   8.77000E+05 sec. 
  40  92 ( 92Zr) 1.62764E+01    0    1.62764E+01  1.00000
  40  91 ( 91Zr) 1.81764E-01    0    1.81764E-01  1.00000
  39  90 ( 90Y ) 2.63291E+01    0    1.25952E+01  0.47837
                                2    1.37340E+01  0.52163   1.14800E+04 sec. 
  39  89 ( 89Y ) 3.05343E+00    0    1.27030E+00  0.41602
                                1    1.78314E+00  0.58398   1.56600E+01 sec. 

  b. Per mass

  A  cross section

  94 1.19960E+00
  93 3.44632E+02
  92 1.39544E+03
  91 1.81764E-01
  90 2.63291E+01
  89 3.05343E+00

Total residual production cross section:  1770.8403320
Non-elastic cross section              :  1770.6293945
\end{verbatim} } \renewcommand{\baselinestretch}{1.07}\small\normalsize
\noindent
At the end of the output, the total calculation time is printed, followed by a 
message that the calculation has been successfully completed:

{\small \begin{verbatim}
Execution time:  0 hours  0 minutes  2.23 seconds 

 The TALYS team congratulates you with this successful calculation.
\end{verbatim} } \renewcommand{\baselinestretch}{1.07}\small\normalsize
\subsubsection{Case b: Discrete state cross sections and spectra}
As a first extension to the simple input/output file given above, we will 
request the 
output of cross sections per individual discrete level.
Also, the cumulated angle-integrated and double-differential 
particle spectra are requested. This is obtained with the following input file:

{\small \begin{verbatim}

#
# n-Nb093-14MeV-spectra
#
# General
#
projectile n
element nb
mass 93
energy 14.
#
# Output
#
outdiscrete y 
outspectra y 
ddxmode 2 
filespectrum n p a 
fileddxa n 30. 
fileddxa n 60. 
fileddxa a 30. 
fileddxa a 60.
\end{verbatim} } \renewcommand{\baselinestretch}{1.07}\small\normalsize
\noindent
In addition to the information printed for case 1a, the cross 
sections per discrete state for each binary channel are given, starting with the
$(n,\gamma )$ channel,

{\small \begin{verbatim}

 5. Binary reactions to discrete levels and continuum

   (n,g)   cross sections:

 Inclusive:

 Level Energy    E-out     J/P       Direct    Compound      Total     Origin

  0   0.00000  21.07609    6.0+     0.00000     0.00000     0.00000    Preeq
  1   0.04089  21.03519    3.0+     0.00000     0.00000     0.00000    Preeq
  2   0.05871  21.01738    4.0+     0.00000     0.00000     0.00000    Preeq
  3   0.07867  20.99742    7.0+     0.00169     0.00000     0.00169    Preeq
...............................
\end{verbatim} } \renewcommand{\baselinestretch}{1.07}\small\normalsize
\noindent
after which the inelastic cross section to every individual discrete state of 
the target nucleus is printed, including the separation in direct and compound, 
see Eq.~(\ref{ineldisc}). These are summed, per contribution, to the total 
discrete inelastic cross section, see Eq.~(\ref{nnprimedisc}). To these cross 
sections, the continuum inelastic cross sections of Eq.~(\ref{nnprimecont}) are 
added to give the total inelastic cross section (\ref{nnprime}). Finally, the
$(n,\gamma n)$ cross section is also printed. This output block looks as follows

{\small \begin{verbatim}

 Inelastic cross sections:

 Inclusive:

 Level Energy    E-out     J/P       Direct    Compound      Total     Origin

  1   0.03077  13.81777    0.5-     0.00000     0.00013     0.00013    Preeq
  2   0.68679  13.16175    1.5-     0.00000     0.00025     0.00025    Preeq
  3   0.74395  13.10459    3.5+     3.27713     0.00049     3.27762    Direct
  4   0.80882  13.03972    2.5+     0.03425     0.00037     0.03463    Direct
  5   0.81032  13.03822    2.5-     2.45130     0.00036     2.45166    Direct
  6   0.94980  12.89874    6.5+     5.68166     0.00066     5.68232    Direct
  7   0.97000  12.87854    1.5-     0.00000     0.00025     0.00025    Preeq
  8   0.97891  12.86963    5.5+     4.86351     0.00063     4.86414    Direct
  9   1.08268  12.76586    4.5+     4.03323     0.00057     4.03379    Direct
 10   1.12709  12.72145    2.5+     0.03389     0.00037     0.03426    Direct
 11   1.28426  12.56428    2.5-     0.03374     0.00035     0.03409    Direct
 12   1.29000  12.55854    0.5-     0.32985     0.00012     0.32997    Direct
 13   1.29722  12.55132    4.5+     0.05613     0.00056     0.05669    Direct
 14   1.31550  12.53304    2.5+     0.00809     0.00036     0.00846    Direct
 15   1.33504  12.51350    8.5+     2.96128     0.00061     2.96189    Direct
                                  ---------   ---------   ---------
 Discrete  Inelastic:              23.76406     0.00609    23.77015
 Continuum Inelastic:             420.88937  1250.63110  1671.52051
                                  ---------   ---------   ---------
 Total     Inelastic:             444.65344  1250.63721  1695.29114

 (n,gn) cross section:     1.59155
\end{verbatim} } \renewcommand{\baselinestretch}{1.07}\small\normalsize
\noindent
This is repeated for the $(n,p)$ and other channels,

{\small \begin{verbatim}

   (n,p)   cross sections:

 Inclusive:

 Level Energy    E-out     J/P       Direct    Compound      Total     Origin

  0   0.00000  14.54008    2.5+     0.33518     0.00026     0.33544    Preeq
  1   0.26682  14.27326    1.5+     1.11920     0.00017     1.11937    Preeq
  2   0.94709  13.59299    0.5+     0.78153     0.00008     0.78161    Preeq
  3   0.94980  13.59028    4.5+     0.08114     0.00037     0.08151    Preeq
...........................
\end{verbatim} } \renewcommand{\baselinestretch}{1.07}\small\normalsize
\noindent
The last column in these tables specifies the origin of the direct contribution
to the discrete state. ``Direct'' means that this is obtained with 
coupled-channels, DWBA or, as in this case, weak coupling, whereas 
``Preeq'' means that the pre-equilibrium cross section is collapsed onto the 
discrete states, as an approximate method for more exact direct reaction 
approaches for charge-exchange and pick-up reactions.
We note here that the feature of calculating, and printing, the inelastic 
cross sections for a specific state is of particular interest in the case of
excitations, i.e. to obtain this particular cross section for a whole range of
incident energies. This will be handled in another sample case.

Since {\bf outspectra y} was specified in the input file, the composite
particle spectra for the continuum are also printed. Besides the total 
spectrum, the division into 
direct (i.e. smoothed collective effects and giant resonance contributions), 
pre-equilibrium, multiple pre-equilibrium and compound is given. First we give 
the photon spectrum,

{\small \begin{verbatim}

 7. Composite particle spectra

 Spectra for outgoing gamma

  Energy   Total       Direct    Pre-equil.  Mult. preeq  Compound

   0.001 2.56193E+00 0.00000E+00 2.56817E-13 0.00000E+00 2.56193E+00
   0.002 3.46083E+00 0.00000E+00 2.44734E-12 0.00000E+00 3.46083E+00
   0.005 6.22649E+00 0.00000E+00 3.40708E-11 0.00000E+00 6.22649E+00
   0.010 1.13850E+01 0.00000E+00 2.57561E-10 0.00000E+00 1.13850E+01
   0.020 2.14177E+01 0.00000E+00 1.94665E-09 0.00000E+00 2.14177E+01
   0.050 5.15819E+01 0.00000E+00 2.71071E-08 0.00000E+00 5.15819E+01
...........................
\end{verbatim} } \renewcommand{\baselinestretch}{1.07}\small\normalsize
\noindent
followed by the neutron spectrum

{\small \begin{verbatim}

Spectra for outgoing neutron

  Energy   Total       Direct    Pre-equil.  Mult. preeq  Compound

   0.001 2.54285E+01 7.90143E-03 9.56260E-02 0.00000E+00 2.53250E+01
   0.002 5.08020E+01 7.91399E-03 1.44361E-01 0.00000E+00 5.06497E+01
   0.005 1.34561E+02 7.95178E-03 2.71935E-01 0.00000E+00 1.34281E+02
   0.010 1.87795E+02 8.01513E-03 4.87956E-01 0.00000E+00 1.87299E+02
   0.020 2.26559E+02 8.14321E-03 9.81582E-01 0.00000E+00 2.25569E+02
...........................
\end{verbatim} } \renewcommand{\baselinestretch}{1.07}\small\normalsize
\noindent
and the spectra for the other outgoing particles.
Depending on the value of {\bf ddxmode}, the double-differential cross 
sections are printed as angular distributions or as spectra per fixed angle.
For the present sample case, {\bf ddxmode 2}, which gives

{\small \begin{verbatim}

 9. Double-differential cross sections per outgoing angle

 DDX for outgoing neutron  at   0.000 degrees

    E-out    Total      Direct     Pre-equil. Mult. preeq   Compound

    0.001 2.02749E+00 1.54078E-03 1.06465E-02 0.00000E+00 2.01530E+00
    0.002 4.04819E+00 1.54323E-03 1.60730E-02 0.00000E+00 4.03057E+00
    0.005 1.07176E+01 1.55060E-03 3.02802E-02 0.00000E+00 1.06857E+01
    0.010 1.49607E+01 1.56295E-03 5.43443E-02 0.00000E+00 1.49048E+01
    0.020 1.80612E+01 1.58792E-03 1.09360E-01 0.00000E+00 1.79502E+01
................................
\end{verbatim} } \renewcommand{\baselinestretch}{1.07}\small\normalsize
\noindent
followed by the other angles and other particles.
A final important feature of the present input file is that some requested 
information has been written to separate output files, i.e. besides the 
standard output file, TALYS also produces the ready-to-plot files

{\small \begin{verbatim}

aspec014.000.tot
nspec014.000.tot
pspec014.000.tot      
\end{verbatim} } \renewcommand{\baselinestretch}{1.07}\small\normalsize
\noindent
containing the angle-integrated neutron, proton and alpha spectra, and
{\small \begin{verbatim}

addx030.0.deg
addx060.0.deg
nddx030.0.deg
nddx060.0.deg
\end{verbatim} } \renewcommand{\baselinestretch}{1.07}\small\normalsize
\noindent
containing the double-differential neutron and alpha spectra at 30 and 60 
degrees.
All these separate output files are in YANDF format, e.g. the (n,xp) emission
spectrum of {\it pspec014.000.tot} looks as follows
{\small \begin{verbatim}

# header:
#   title: Nb93(n,xp) emission spectrum at  1.400000E+01 MeV
#   source: TALYS-2.0
#   user: Arjan Koning
#   date: 2023-12-12
#   format: YANDF-0.1
# target:
#   Z: 41
#   A: 93
#   nuclide: Nb93
# reaction:
#   type: (n,xp)
#   ENDF_MF: 6
#   ENDF_MT: 5
#   E-incident [MeV]:  1.400000E+01
#   E-average [MeV]:  8.456717E+00
# datablock:
#   quantity: emission spectrum
#   columns: 7
#   entries: 151
##     E-out           xs           Direct     Preequilibrium Multiple preeq
##     [MeV]        [mb/MeV]       [mb/MeV]       [mb/MeV]       [mb/MeV]   
   2.000000E-01   2.678184E-30   0.000000E+00   2.678184E-30   0.000000E+00 
   3.000000E-01   3.599347E-23   0.000000E+00   3.599347E-23   0.000000E+00 
   4.000000E-01   5.176781E-19   0.000000E+00   5.176781E-19   0.000000E+00 
   5.000000E-01   3.460652E-16   0.000000E+00   3.460652E-16   0.000000E+00 
.............................
\end{verbatim} } \renewcommand{\baselinestretch}{1.07}\small\normalsize
\noindent

\subsubsection{Case c: Exclusive channels and spectra}
As another extension of the simple input file we can print the exclusive cross
sections at one incident energy and the associated exclusive spectra.
This is accomplished with the input file

{\small \begin{verbatim}
#
# n-Nb093-14MeV-general-exclusive
#
# General
#
projectile n
element nb
mass 93
energy 14.
#
# Output
#
channels y 
outspectra y
\end{verbatim} } \renewcommand{\baselinestretch}{1.07}\small\normalsize
\noindent
Contrary to the previous sample case, in this case no double-differential 
cross sections or 
results per separate file are printed (since it only concerns one incident 
energy). The exclusive cross sections are given 
in one table, per channel and per ground or isomeric state. It is checked 
whether the exclusive cross sections add up to the non-elastic cross section. 
Note that this sum rule, Eq.~(\ref{prod}), is only expected to hold if we 
include 
enough exclusive channels in the calculation. If {\bf maxchannel 4}, this 
equality should always hold for incident energies up to 20 MeV. This output 
block looks as follows:

{\small \begin{verbatim}

 6. Exclusive cross sections

 6a. Total exclusive cross sections

     Emitted particles     cross section reaction         level    isomeric   
    n   p   d   t   h   a                                        cross section
    0   0   0   0   0   0    1.19960E+00  (n,g)                               
                                                             0     5.71296E-01
                                                             1     6.28304E-01
    1   0   0   0   0   0    3.13448E+02  (n,n')                              
                                                             0     2.76520E+02
                                                             1     3.69274E+01
    0   1   0   0   0   0    2.97658E+01  (n,p)                               
    0   0   1   0   0   0    4.96449E+00  (n,d)                               
    0   0   0   1   0   0    1.81764E-01  (n,t)                               
    0   0   0   0   0   1    2.63285E+01  (n,a)                               
                                                             0     1.25948E+01
                                                             2     1.37337E+01
    2   0   0   0   0   0    1.37899E+03  (n,2n)                              
                                                             0     8.65528E+02
                                                             1     5.13465E+02
    1   1   0   0   0   0    1.13115E+01  (n,np)                              
    1   0   0   0   0   1    3.05339E+00  (n,na)                              
                                                             0     1.27026E+00
                                                             1     1.78312E+00
    0   1   0   0   0   1    5.03111E-06  (n,pa)                              

 Absorption cross section                 :    62.44013

 Sum over exclusive channel cross sections:  1769.24536
 (n,gn) + (n,gp) +...(n,ga) cross sections:     1.59452
 Total                                    :  1770.83984
 Non-elastic cross section                :  1770.62939
\end{verbatim} } \renewcommand{\baselinestretch}{1.07}\small\normalsize
\noindent
Note that the $(n,np)$ and $(n,d)$ cross sections add up to the residual 
production cross section for ${}^{92}$Zr, as given in the first sample case.

Since {\bf outspectra y}, for each exclusive channel the spectrum per 
outgoing particle is given. This output block begins with:

{\small \begin{verbatim}

 6b. Exclusive spectra

      Emitted particles     cross section reaction      gamma cross section
    n   p   d   t   h   a
    1   0   0   0   0   0    3.26664E+02  (n,n')            1.00870E+03

  Outgoing spectra

  Energy  gamma       neutron     proton      deuteron    triton      helium-3  

 0.001 4.06619E+00 2.14250E-03 0.00000E+00 0.00000E+00 0.00000E+00 0.00000E+00
 0.002 4.08311E+00 4.22497E-03 0.00000E+00 0.00000E+00 0.00000E+00 0.00000E+00
 0.005 4.13354E+00 1.04500E-02 0.00000E+00 0.00000E+00 0.00000E+00 0.00000E+00
 0.010 4.21777E+00 2.08288E-02 0.00000E+00 0.00000E+00 0.00000E+00 0.00000E+00
 0.020 4.38694E+00 4.16553E-02 0.00000E+00 0.00000E+00 0.00000E+00 0.00000E+00
 0.050 4.92328E+00 1.04608E-01 0.00000E+00 0.00000E+00 0.00000E+00 0.00000E+00
 0.100 6.46225E+00 2.10358E-01 0.00000E+00 0.00000E+00 0.00000E+00 0.00000E+00
..................................
      Emitted particles     cross section reaction      gamma cross section
    n   p   d   t   h   a
    1   1   0   0   0   0    1.06861E+01  (n,np)            1.97965E+01
 
  Outgoing spectra
 
  Energy  gamma       neutron     proton      deuteron    triton      helium-3  
   
 0.001 6.35050E-02 1.87485E-01 0.00000E+00 0.00000E+00 0.00000E+00 0.00000E+00
 0.002 6.94942E-02 3.74836E-01 0.00000E+00 0.00000E+00 0.00000E+00 0.00000E+00
 0.005 8.74445E-02 9.36836E-01 0.00000E+00 0.00000E+00 0.00000E+00 0.00000E+00
 0.010 1.17368E-01 1.88679E+00 0.00000E+00 0.00000E+00 0.00000E+00 0.00000E+00
 0.020 1.77238E-01 3.89497E+00 0.00000E+00 0.00000E+00 0.00000E+00 0.00000E+00
 0.050 3.57037E-01 6.03595E+00 0.00000E+00 0.00000E+00 0.00000E+00 0.00000E+00
 0.100 6.57333E-01 8.96107E+00 0.00000E+00 0.00000E+00 0.00000E+00 0.00000E+00
 0.200 1.58990E+00 1.04045E+01 0.00000E+00 0.00000E+00 0.00000E+00 0.00000E+00
 0.300 1.88712E+00 9.88185E+00 0.00000E+00 0.00000E+00 0.00000E+00 0.00000E+00
 0.400 2.01034E+00 9.03657E+00 0.00000E+00 0.00000E+00 0.00000E+00 0.00000E+00
 0.500 1.41701E+00 8.28305E+00 0.00000E+00 0.00000E+00 0.00000E+00 0.00000E+00
 0.600 5.68786E+00 7.59201E+00 0.00000E+00 0.00000E+00 0.00000E+00 0.00000E+00
 0.700 8.64822E+00 6.60600E+00 0.00000E+00 0.00000E+00 0.00000E+00 0.00000E+00
 0.800 4.28359E+00 5.86945E+00 0.00000E+00 0.00000E+00 0.00000E+00 0.00000E+00
 0.900 2.37448E+00 5.09628E+00 0.00000E+00 0.00000E+00 0.00000E+00 0.00000E+00
 1.000 2.20641E+00 4.19745E+00 1.23263E-07 0.00000E+00 0.00000E+00 0.00000E+00
..................................
\end{verbatim} } \renewcommand{\baselinestretch}{1.07}\small\normalsize
\noindent
Note, as explained in Section \ref{exclspec}, that the (n,np) channel is 
characterized by both a neutron and a proton spectrum.
\subsubsection{Case d: Nuclear structure}    
It is possible to have all the nuclear structure information in the output file.
The simplest way is to set {\bf outbasic y}, which means that about everything 
that can be printed, will be printed. This may be a bit overdone if one
is only interested in e.g. discrete levels or level densities. If the 
keywords {\bf outlevels} and/or {\bf outdensity} are set to {\bf y}, 
discrete level and level density information will always be given for the 
target nucleus and the primary compound nucleus. With {\bf outgamma y}, photon
strength function information is also given. If we would set, in addition,
{\bf outpopulation y}, this info will also be given for all the other residual 
nuclides that are reached in the reaction chain. 
The input file for this sample case is

{\small \begin{verbatim}

#
# General
#
projectile n
element nb
mass 93
energy 14.
#
# Output
#
outlevels y
outdensity y
outgamma y
\end{verbatim} } \renewcommand{\baselinestretch}{1.07}\small\normalsize
\noindent
In addition to the output of case 1a,
the separation energies for the six light particles are printed. 

{\small \begin{verbatim}

 NUCLEAR STRUCTURE INFORMATION FOR Z= 41 N= 52 ( 93Nb)

 Mass in a.m.u.     :  92.906373

 Separation energies:

 Particle        S

 neutron      8.83087
 proton       6.04279
 deuteron    12.45297
 triton      13.39009
 helium-3    15.64713
 alpha        1.92655
\end{verbatim} } \renewcommand{\baselinestretch}{1.07}\small\normalsize
\noindent
In the next output block,
the discrete level scheme is printed for the first levels. 
The discrete level info contains level number, energy, spin, parity, branching 
ratios and lifetimes of possible isomers. It is also indicated whether the 
spin (J) or parity (P) of a level is experimentally known or whether a value 
was assigned to it (see Section \ref{disclevel}). The ``string'' of the 
original ENSDF database is also given, so that the user can learn about 
possible alternative choices for spin and parity. This output block begins with:

{\small \begin{verbatim}

 Discrete levels of Z= 41 N= 52 ( 93Nb)

 Number  Energy Spin Parity  Branching Ratio (%) Lifetime(sec) Assignment        ENSDF

   0     0.0000  4.5   +                                                 
   1     0.0308  0.5   -                           5.087E+08            
                               --->  0  100.0000
   2     0.6868  1.5   -                                    
                               --->  1  100.0000
   3     0.7440  3.5   +                                   
                               --->  0  100.0000
   4     0.8088  2.5   +                                  
                               --->  3    2.1580
                               --->  0   97.8420
....................................
\end{verbatim} } \renewcommand{\baselinestretch}{1.07}\small\normalsize
\noindent
Since {\bf outdensity y}, we print all the level density parameters that are 
involved in the main output file.
Note also that, since {\bf outdensity y} by default implies {\bf filedensity y},
the files {\it ld041093.tot} and  {\it ld041094.tot} have been created in YANDF format. They
contain all level density parameters and a comparison between cumulated 
discrete levels and the integrated level density.
discussed in Section \ref{leveldens}: the level density parameter at the 
neutron separation energy $a(S_{n})$, the experimental and theoretical average 
resonance spacing $D_{0}$, the asymptotic level density parameter 
$\tilde{a}$, the shell damping parameter $\gamma$, the pairing energy $\Delta$, 
the shell correction energy $\delta W$, the matching energy $E_{x}$, the last 
discrete level, the levels for the matching problem, the temperature $T$, the 
back-shift energy $E_{0}$, the discrete state spin cut-off parameter 
$\sigma$ and the spin cut-off parameter at the neutron separation energy.
Next, we print a table with the level density parameter $a$, the spin cut-off 
parameter and the level density itself, all as a function of the excitation 
energy. The file {\bf ld041093.tot} looks as follows:

{\small \begin{verbatim}

# header:
#   title: Nb93 level density
#   source: TALYS-2.0
#   user: Arjan Koning
#   date: 2023-12-12
#   format: YANDF-0.1
# residual:
#   Z: 41
#   A: 93
#   nuclide: Nb93
# parameters:
#   ldmodel keyword: 1
#   level density model: Gilbert-Cameron
#   Collective enhacement: n
#   a(Sn) [MeV^-1]:  1.233156E+01
#   asymptotic a [MeV^-1]:  1.224515E+01
#   shell correction [MeV]:  1.037895E-01
#   damping gamma:  9.559104E-02
#   pairing energy [MeV]:  1.244342E+00
#   adjusted pairing shift [MeV]:  0.000000E+00
#   separation energy [MeV]:  8.830873E+00
#   discrete spin cutoff parameter:  1.165425E+01
#   spin cutoff parameter(Sn):  2.095789E+01
#   matching energy [MeV]:  7.615608E+00
#   temperature [MeV]:  8.670035E-01
#   E0 [MeV]: -1.345862E+00
#   Nlow: 5
#   Ntop: 15
#   ctable:  1.000000E-20
#   ptable:  1.000000E-20
# observables:
#   experimental D0 [eV]:  0.000000E+00
#   experimental D0 unc. [eV]:  0.000000E+00
#   theoretical D0 [eV]:  7.795445E+01
#   Chi-2 D0:  0.000000E+00
#   C/E D0:  0.000000E+00
#   Frms D0:  0.000000E+00
#   Erms D0:  0.000000E+00
#   Chi-2 per level:  1.128847E-01
#   Frms per level:  1.030749E+00
#   Erms per level:  1.005810E+00
#   average deviation per level:  7.030114E-02
# datablock:
#   quantity: level density
#   columns: 6
#   entries: 100
##       E            Level      N_cumulative     Total LD           a       
##     [MeV]           []             []          [MeV^-1]       [MeV^-1]    
   9.498000E-01     6            7.096463E+00   1.628955E+01   1.236664E+01 
   9.700000E-01     7            7.429368E+00   1.667353E+01   1.236664E+01 
   9.789100E-01     8            7.578694E+00   1.684576E+01   1.236664E+01 
   1.082680E+00     9            9.434585E+00   1.898762E+01   1.236664E+01 
   1.127090E+00    10            1.029970E+01   1.998556E+01   1.236664E+01 
   1.284260E+00    11            1.373884E+01   2.395769E+01   1.236641E+01 
   1.290000E+00    12            1.387682E+01   2.411683E+01   1.236638E+01
....................................
\end{verbatim} } \renewcommand{\baselinestretch}{1.07}\small\normalsize
\noindent

With {\bf outgamma y} the gamma-ray information is printed. 
First, all relevant parameters are given: the total radiative width 
$\Gamma_{\Gamma}$, the s-wave resonance spacing $D_{0}$, 
the s-wave strength function $S_{0}$ and the normalization factor for the
gamma-ray strength function. Second, we print the giant resonance information.
For each multipolarity, we print the strength of the giant resonance
$\sigma_{0}$, its energy and its width.
Next, the gamma-ray strength function and transmission coefficients for this 
multipolarity and as a function of energy are printed.
This output block begins with:

{\small \begin{verbatim}

########## GAMMA STRENGTH FUNCTIONS, TRANSMISSION COEFFICIENTS AND CROSS SEC

 Gamma-ray information for Z= 41 N= 53 ( 94Nb)

 S-wave strength function parameters:

 Exp. total radiative width=   0.14500 eV +/- 0.01000 Theor. total radiative 
                                                      Theor. total radiative 
                                                      Theor. total radiative 
 Exp. D0                   =     80.00 eV +/-   10.00 Theor. D0              
                                                      Theor. D1              
 Theor. S-wave strength f. =  15.47386E-4
 Average resonance energy  =     10000.00 eV

 Incident energy: E[MeV]=  14.000

 Gamma-ray strength function model for E1: IAEA-CRP SMLO 2019 Tables

 Gamma-ray strength function model for M1: IAEA GSF CRP (2018)

 Adjustable parameters for E1: etable=   0.00000 ftable=   1.00000 wtable=   

 Inclusion of an E1 upbend C x U*/ (1+exp(E-eta)) with  C=  0.00E+00 eta=   

 Inclusion of an M1 upbend C exp(-F*|beta2|) exp(-eta*E) with C=  3.50E-0

 Normalized gamma-ray strength functions and transmission coefficients for l= 1

 Giant resonance parameters :

 sigma0(M1) =   1.322       sigma0(E1) = 192.148    PR: sigma0(M1) =   0.066 
      E(M1) =   8.441            E(E1) =  16.523    PR:      E(M1) =   3.174 
  gamma(M1) =   4.000        gamma(E1) =   5.515    PR:  gamma(M1) =   1.500 
      k(M1) =   8.67373E-08      k(E1) =   8.67373E-08

      E       f(M1)        f(E1)        T(M1)        T(E1)

    0.001  1.79672E-08  1.56975E-09  1.12891E-16  2.65817E-17
    0.002  1.79538E-08  1.57089E-09  9.02457E-16  2.12748E-16
    0.005  1.79122E-08  1.57435E-09  1.40682E-14  3.32861E-15
    0.010  1.78432E-08  1.58011E-09  1.12112E-13  2.66879E-14
    0.020  1.77059E-08  1.59153E-09  8.89998E-13  2.14444E-13
    0.050  1.73010E-08  1.62634E-09  1.35881E-11  3.39519E-12
....................................
\end{verbatim} } \renewcommand{\baselinestretch}{1.07}\small\normalsize
\noindent
which is repeated for each $l$-value. Finally, the photoabsorption
cross section is printed:

{\small \begin{verbatim}

Photoabsorption cross sections

  E [MeV]   xs [mb]

   1.000000E-03   2.559201E-04
   2.000000E-03   6.118469E-04
   5.000000E-03   1.377138E-03
   1.000000E-02   2.647125E-03
   2.000000E-02   5.166643E-03
   5.000000E-02   1.256542E-02
....................................
\end{verbatim} } \renewcommand{\baselinestretch}{1.07}\small\normalsize
Since we put in this input file the keyword {\bf filepsf y} detailed
PSF information is also available in separate output files. For example
{\bf psf041094.E1} reads as follows:
{\small \begin{verbatim}
# header:
#   title: Nb94 photon strength function
#   source: TALYS-2.0
#   user: Arjan Koning
#   date: 2023-12-12
#   format: YANDF-0.1
# residual:
#   Z: 41
#   A: 94
#   nuclide: Nb94
# parameters:
#   strength keyword: 9
#   PSF model: IAEA-CRP SMLO 2019 Tables
#   radiation type: E1
# observables:
#   experimental Gamma_gamma [eV]:  1.450000E-01
#   experimental Gamma_gamma unc. [eV]:  1.000000E-02
#   theoretical Gamma_gamma [eV]:  1.236158E-01
#   C/E Gamma_gamma:  8.525231E-01
# datablock:
#   quantity: photon strength function
#   columns: 2
#   entries: 86
##       E            f(E1)
##     [MeV]           []
   1.000000E-03   1.569746E-09
   2.000000E-03   1.570894E-09
   5.000000E-03   1.574351E-09
   1.000000E-02   1.580114E-09
   2.000000E-02   1.591526E-09
.........................................
\end{verbatim} } \renewcommand{\baselinestretch}{1.07}\small\normalsize

\subsubsection{Case e: Detailed pre-equilibrium information}
The single- and double-differential spectra have already been covered in 
sample 1b. In addition to this, the contribution of the pre-equilibrium 
mechanism to the spectra 
and cross sections can be printed in more detail with the 
{\bf outpreequilibrium} keyword.
With the input file

{\small \begin{verbatim}

#
# n-Nb093-14MeV-preeq
#
# General
#
projectile n
element nb
mass 93
energy 14.
#
# Output
#
outpreequilibrium y 
outspectra y
ddxmode 2
\end{verbatim} } \renewcommand{\baselinestretch}{1.07}\small\normalsize
\noindent
we obtain, in addition to the aforementioned output blocks, a detailed outline
of the pre-equilibrium model used, in this case the default: the two-component 
exciton model.
First, the parameters for the exciton model are printed, followed by
the matrix elements as a function of the exciton number:

{\small \begin{verbatim}

 ########## PRE-EQUILIBRIUM ##########                      
                     
 ++++++++++ TWO-COMPONENT EXCITON MODEL ++++++++++
                               
 1. Matrix element for E=  21.076        
                                        
 Constant for matrix element :   1.000                                          
 p-p ratio for matrix element:   1.000
 n-n ratio for matrix element:   1.500                                          
 p-n ratio for matrix element:   1.000
 n-p ratio for matrix element:   1.000
 
 p(p) h(p) p(n) h(n)     M2pipi      M2nunu      M2pinu      M2nupi
 
  0    0    1    0    2.63420E-05 3.95131E-05 2.63420E-05 2.63420E-05
  0    0    2    1    1.08884E-04 1.63327E-04 1.08884E-04 1.08884E-04
  1    1    1    0    1.08884E-04 1.63327E-04 1.08884E-04 1.08884E-04
  0    0    3    2    1.76643E-04 2.64964E-04 1.76643E-04 1.76643E-04
  1    1    2    1    1.76643E-04 2.64964E-04 1.76643E-04 1.76643E-04
.....................................
\end{verbatim} } \renewcommand{\baselinestretch}{1.07}\small\normalsize
\noindent
Next, the emission 
rates are printed: first as function of particle type and particle-hole number,
and in the last column summed over particles:

{\small \begin{verbatim}

 2. Emission rates or escape widths

 A. Emission rates ( /sec)

 p(p) h(p) p(n) h(n)    gamma       neutron     proton      deuteron    triton  

 0    0    1    0    2.16570E+18 0.00000E+00 0.00000E+00 0.00000E+00 0.00000E+0
 0    0    2    1    5.29285E+17 1.63578E+21 0.00000E+00 0.00000E+00 0.00000E+0
 1    1    1    0    5.58040E+17 8.15540E+20 2.32728E+20 2.79804E+19 0.00000E+0
 0    0    3    2    1.26381E+17 4.52770E+20 0.00000E+00 0.00000E+00 0.00000E+0
 1    1    2    1    1.08464E+17 3.04371E+20 1.11136E+19 3.36637E+18 9.21406E+1
.....................................
\end{verbatim} } \renewcommand{\baselinestretch}{1.07}\small\normalsize
\noindent
Also, the alternative representation in terms of the escape widths, see
Eqs.~(\ref{gamesc}) and ~(\ref{gamesct}), is given,

{\small \begin{verbatim}

 B. Escape widths (MeV)

p(p) h(p) p(n) h(n)    gamma       neutron     proton      deuteron    triton  

0    0    1    0    1.42549E-03 0.00000E+00 0.00000E+00 0.00000E+00 0.00000E+0
0    0    2    1    3.48382E-04 1.07669E+00 0.00000E+00 0.00000E+00 0.00000E+0
1    1    1    0    3.67309E-04 5.36798E-01 1.53184E-01 1.84170E-02 0.00000E+0
0    0    3    2    8.31858E-05 2.98019E-01 0.00000E+00 0.00000E+00 0.00000E+0
1    1    2    1    7.13921E-05 2.00341E-01 7.31511E-03 2.21578E-03 6.06481E-0
2    2    1    0    9.92804E-05 8.90114E-02 1.26899E-02 1.26325E-03 0.00000E+0
.....................................
\end{verbatim} } \renewcommand{\baselinestretch}{1.07}\small\normalsize
\noindent
The internal transition rates such as those of Eq.~(\ref{lambdapiplus}) and 
the associated damping and total widths are given next, 

{\small \begin{verbatim}

 3. Internal transition rates or damping widths, total widths

 A. Internal transition rates ( /sec)

 p(p) h(p) p(n) h(n)     lambdapiplus   lambdanuplus    lambdapinu     lambdanup

0    0    1    0       1.45456E+21    1.74977E+21    0.00000E+00    0.00000E+0
0    0    2    1       3.18197E+21    3.67047E+21    0.00000E+00    1.62232E+2
1    1    1    0       1.74347E+21    3.78846E+21    1.30118E+20    0.00000E+0
0    0    3    2       3.26507E+21    3.59251E+21    0.00000E+00    7.12888E+2
1    1    2    1       2.35211E+21    3.64486E+21    1.72552E+20    2.29011E+2
.....................................
\end{verbatim} } \renewcommand{\baselinestretch}{1.07}\small\normalsize
\noindent
The lifetimes, $t(p,h)$ of Eq.~(\ref{lifetime}) and the depletion factors 
$D_{p,h}$ of Eq.~(\ref{depletion}), are printed next, 

{\small \begin{verbatim}

 4. Lifetimes
 p(p) h(p) p(n) h(n)      Strength

  0    0    1    0       3.11867E-22
  0    0    2    1       6.41109E-23
  1    1    1    0       6.88639E-23
  0    0    3    2       3.09624E-23
  1    1    2    1       7.50433E-23
  2    2    1    0       2.43181E-23
.....................................
\end{verbatim} } \renewcommand{\baselinestretch}{1.07}\small\normalsize
\noindent
The partial state densities are printed for the first particle-hole 
combinations as a function of excitation energy. We also print the exciton 
number-dependent spin distributions and their sum, to see whether we have 
exhausted all spins. This output block is as follows

{\small \begin{verbatim}

 ++++++++++ PARTIAL STATE DENSITIES ++++++++++

 Particle-hole state densities

    Ex   P(n=3)   gp     gn                            Configuration p(p) h(p) 
                              1 1 0 0   0 0 1 1   1 1 1 0   1 0 1 1   2 1 0 0  

 1.000  0.000  2.733  3.533 7.471E+00 1.248E+01 9.728E+00 1.139E+01 0.000E+00 
 2.000  0.000  2.733  3.533 1.494E+01 2.497E+01 4.559E+01 5.633E+01 8.212E+00 
 3.000  0.000  2.733  3.533 2.241E+01 3.745E+01 1.078E+02 1.354E+02 2.627E+01 
 4.000  0.000  2.733  3.533 2.988E+01 4.994E+01 1.965E+02 2.486E+02 5.453E+01 
 5.000  0.000  2.733  3.533 3.736E+01 6.242E+01 3.116E+02 3.959E+02 9.301E+01 
 6.000  0.000  2.733  3.533 4.483E+01 7.491E+01 4.530E+02 5.774E+02 1.417E+02 
.....................................
 Particle-hole spin distributions

  n     J= 0        J= 1        J= 2        J= 3        J= 4        J= 5       
 
  1  1.7785E-02  4.3554E-02  4.8369E-02  3.6832E-02  2.1025E-02  9.3136E-03  
  2  6.3683E-03  1.7261E-02  2.3483E-02  2.4247E-02  2.0773E-02  1.5284E-02  
  3  3.4812E-03  9.7603E-03  1.4208E-02  1.6237E-02  1.5926E-02  1.3878E-02  
  4  2.2659E-03  6.4613E-03  9.7295E-03  1.1698E-02  1.2277E-02  1.1642E-02  
  5  1.6234E-03  4.6764E-03  7.1862E-03  8.9070E-03  9.7354E-03  9.7128E-03  
.....................................
\end{verbatim} } \renewcommand{\baselinestretch}{1.07}\small\normalsize
\noindent
We print a table with the pre-equilibrium cross sections per stage and outgoing 
energy, for each outgoing particle. At the end of each table, we give the total
pre-equilibrium cross sections per particle. Finally the total pre-equilibrium 
cross section summed over outgoing particles is printed,

{\small \begin{verbatim}

 ++++++++++ TOTAL PRE-EQUILIBRIUM CROSS SECTIONS ++++++++++
  
 Pre-equilibrium cross sections for gamma
  
    E      Total        p=1        p=2        p=3        p=4        p=5        

  0.001 2.2249E-13 9.1810E-14 5.0384E-14 3.8551E-14 4.1750E-14 0.0000E+00 0.000
  0.002 2.1767E-12 8.9578E-13 4.9358E-13 3.7794E-13 4.0941E-13 0.0000E+00 0.000
  0.005 2.9926E-11 1.2218E-11 6.8126E-12 5.2280E-12 5.6681E-12 0.0000E+00 0.000
  0.010 2.2566E-10 9.0941E-11 5.1690E-11 3.9811E-11 4.3221E-11 0.0000E+00 0.000
  0.020 1.7108E-09 6.7253E-10 3.9631E-10 3.0739E-10 3.3461E-10 0.0000E+00 0.000
  0.050 2.4211E-08 8.9017E-09 5.7581E-09 4.5550E-09 4.9961E-09 0.0000E+00 0.000
  0.100 1.7108E-07 5.7425E-08 4.1833E-08 3.4050E-08 3.7774E-08 0.0000E+00 0.000
.....................................
 19.000 1.3505E-01 1.3040E-01 4.3802E-03 2.6369E-04 4.8448E-06 0.0000E+00 0.000
 19.500 1.2227E-01 1.1875E-01 3.4007E-03 1.2338E-04 6.6196E-07 0.0000E+00 0.000
 20.000 1.1171E-01 1.0895E-01 2.7321E-03 3.1776E-05 0.0000E+00 0.0000E+00 0.000
 21.000 0.0000E+00 0.0000E+00 0.0000E+00 0.0000E+00 0.0000E+00 0.0000E+00 0.000
 22.000 0.0000E+00 0.0000E+00 0.0000E+00 0.0000E+00 0.0000E+00 0.0000E+00 0.000

         1.2864E+00 1.1291E+00 1.2096E-01 2.6279E-02 1.0140E-02 0.0000E+00 0.000

 Integrated:     1.28644


 Pre-equilibrium cross sections for neutron

    E      Total        p=1        p=2        p=3        p=4        p=5        

  0.001 9.4598E-02 0.0000E+00 2.6248E-02 2.5416E-02 2.0231E-02 1.4060E-02 8.642
  0.002 1.4281E-01 0.0000E+00 3.9637E-02 3.8373E-02 3.0539E-02 2.1220E-02 1.304
  0.005 2.6901E-01 0.0000E+00 7.4728E-02 7.2301E-02 5.7513E-02 3.9941E-02 2.453
  0.010 4.8271E-01 0.0000E+00 1.3428E-01 1.2979E-01 1.0316E-01 7.1572E-02 4.391
  0.020 9.7103E-01 0.0000E+00 2.7090E-01 2.6131E-01 2.0733E-01 1.4358E-01 8.790
.....................................
 17.500 2.5293E-01 0.0000E+00 0.0000E+00 0.0000E+00 0.0000E+00 0.0000E+00 0.000
 18.000 0.0000E+00 0.0000E+00 0.0000E+00 0.0000E+00 0.0000E+00 0.0000E+00 0.000
 18.500 0.0000E+00 0.0000E+00 0.0000E+00 0.0000E+00 0.0000E+00 0.0000E+00 0.000
 19.000 0.0000E+00 0.0000E+00 0.0000E+00 0.0000E+00 0.0000E+00 0.0000E+00 0.000

        2.6727E+01 0.0000E+00 0.0000E+00 0.0000E+00 1.0326E-02 1.1088E-02 4.974

 Integrated:    26.72690

 Total pre-equilibrium cross section:   415.82114
\end{verbatim} } \renewcommand{\baselinestretch}{1.07}\small\normalsize
\end{samplecase}
\begin{samplecase}
\subsubsection{Case f: Discrete direct cross sections and angular 
distributions}
More specific information on the characteristics of direct reactions can be 
obtained with the following input file,

{\small \begin{verbatim}

#
# n-Nb093-14MeV-discrete
#
# General
#
projectile n
element nb
mass 93
energy 14.
#
# Output
#
outdiscrete         y
outangle            y     
outlegendre         y     
outdirect           y
outspectra          y
\end{verbatim} } \renewcommand{\baselinestretch}{1.07}\small\normalsize
\noindent
Now we obtain, through {\bf outdirect y}, the direct cross sections from 
inelastic collective scattering and giant resonances. The output block
begins with

{\small \begin{verbatim}

 ++++++++++ DIRECT CROSS SECTIONS ++++++++++

 Direct inelastic cross sections

 Level  Energy   E-out      J/P   Cross section  Def. par.

   3   0.74395  13.10459    3.5+     3.27713     B  0.04108
   4   0.80882  13.03972    2.5+     0.03425     B  0.00586
   5   0.81032  13.03822    2.5-     2.45130     B  0.03558
   6   0.94980  12.89874    6.5+     5.68166     B  0.05434
   8   0.97891  12.86963    5.5+     4.86351     B  0.05031
   9   1.08268  12.76586    4.5+     4.03323     B  0.04593
.....................................
  
 Discrete direct inelastic cross section:    23.76406   Level 1- 30
 Collective cross section in continuum  :    32.30161
\end{verbatim} } \renewcommand{\baselinestretch}{1.07}\small\normalsize
\noindent
which for the case of ${}^{93}$Nb gives the results of the weak-coupling 
model. For every level, the angular distribution is given, since
{\bf outangle y} was specified:

{\small \begin{verbatim}

 Direct inelastic angular distributions

 Angle Ex= 0.744   Ex= 0.809   Ex= 0.810   Ex= 0.950   Ex= 0.979   Ex= 1.083   
        JP= 3.5+    JP= 2.5+    JP= 2.5-    JP= 6.5+    JP= 5.5+    JP= 4.5+   

   0.0 1.00425E+00 1.58216E-03 7.50804E-01 1.73829E+00 1.48762E+00 1.23259E+00 
   2.0 1.00116E+00 1.59623E-03 7.48488E-01 1.73292E+00 1.48303E+00 1.22878E+00 
   4.0 9.92237E-01 1.63798E-03 7.41812E-01 1.71744E+00 1.46977E+00 1.21779E+00 
   6.0 9.78514E-01 1.70633E-03 7.31535E-01 1.69357E+00 1.44933E+00 1.20082E+00 
.....................................
\end{verbatim} } \renewcommand{\baselinestretch}{1.07}\small\normalsize
\noindent
The table with total giant resonance results is given next,

{\small \begin{verbatim}

 ++++++++++ GIANT RESONANCES ++++++++++

      Cross section   Exc. energy Emis. energy   Width    Deform. par.

 GMR  :     0.00000    16.37500    -2.52646     3.00000     0.02645
 GQR  :     0.00000    14.34671    -0.49817     4.14092     0.14236
 LEOR :    24.61940     6.84228     7.00626     5.00000     0.15971
 HEOR :     0.00000    25.38264   -11.53410     7.36250     0.13210

 Total:    24.61940
\end{verbatim} } \renewcommand{\baselinestretch}{1.07}\small\normalsize
\noindent
followed, since {\bf outspectra y}, by the associated spectra,

{\small \begin{verbatim}

 Giant resonance spectra

  Energy   Total       GMR        GQR       LEOR       HEOR     Collective

   0.001 7.7214E-03 0.0000E+00 0.0000E+00 7.7214E-03 0.0000E+00 0.0000E+00
   0.002 7.7337E-03 0.0000E+00 0.0000E+00 7.7337E-03 0.0000E+00 0.0000E+00
   0.005 7.7706E-03 0.0000E+00 0.0000E+00 7.7706E-03 0.0000E+00 0.0000E+00
   0.010 7.8325E-03 0.0000E+00 0.0000E+00 7.8325E-03 0.0000E+00 0.0000E+00
   0.020 7.9577E-03 0.0000E+00 0.0000E+00 7.9577E-03 0.0000E+00 0.0000E+00
   0.050 8.3441E-03 0.0000E+00 0.0000E+00 8.3441E-03 0.0000E+00 0.0000E+00
.....................................
\end{verbatim} } \renewcommand{\baselinestretch}{1.07}\small\normalsize
\noindent
The total, i.e. direct $+$ compound cross section per discrete level of 
each residual nucleus was already described for sample 1b. In addition, we have
now requested the angular distributions and the associated Legendre 
coefficients. 
First, the angular distribution for elastic scattering, separated by direct and 
compound contribution, is given. Since {\bf outlegendre y} it is given first in 
terms of Legendre coefficients. This output block begins with:

{\small \begin{verbatim}

 8. Discrete state angular distributions

 8a1. Legendre coefficients for elastic scattering

   L       Total           Direct         Compound       Normalized

   0     1.75971E+02     1.75971E+02     4.71035E-05     7.95776E-02
   1     1.55728E+02     1.55728E+02     0.00000E+00     7.04232E-02
   2     1.39720E+02     1.39720E+02     1.86062E-06     6.31840E-02
   3     1.21488E+02     1.21488E+02     0.00000E+00     5.49390E-02
   4     1.02659E+02     1.02659E+02     3.25580E-07     4.64241E-02
.....................................
\end{verbatim} } \renewcommand{\baselinestretch}{1.07}\small\normalsize
\noindent
where the final column means division of the Legendre coefficients by the
cross section.
This is followed by the associated angular distribution. This output block 
begins with:

{\small \begin{verbatim}

 8a2. Elastic scattering angular distribution

 Angle        Total          Direct         Compound

   0.0     6.69554E+03     6.69554E+03     6.12260E-05
   2.0     6.62134E+03     6.62134E+03     6.11570E-05
   4.0     6.40317E+03     6.40317E+03     6.09526E-05
   6.0     6.05390E+03     6.05390E+03     6.06204E-05
   8.0     5.59369E+03     5.59369E+03     6.01725E-05
  10.0     5.04824E+03     5.04824E+03     5.96247E-05
  12.0     4.44657E+03     4.44657E+03     5.89949E-05
.....................................
\end{verbatim} } \renewcommand{\baselinestretch}{1.07}\small\normalsize
\noindent
Next, the Legendre coefficients for inelastic scattering to each discrete
level, separated by the direct and 
compound contribution, is given. This output block begins with:

{\small \begin{verbatim}

 8b1. Legendre coefficients for inelastic scattering
  
    Level  1
  
   L       Total           Direct         Compound       Normalized
  
   0     5.43660E-04     5.33254E-04     1.04057E-05     7.95775E-02
   1     1.86109E-04     1.86109E-04     0.00000E+00     2.72415E-02
   2     6.20916E-05     6.21244E-05    -3.27812E-08     9.08857E-03
   3     1.66523E-05     1.66523E-05     0.00000E+00     2.43746E-03
   4    -1.37214E-05    -1.35853E-05    -1.36110E-07    -2.00846E-03
   5    -1.90397E-05    -1.90397E-05     0.00000E+00    -2.78691E-03
   6    -1.61471E-05    -1.61584E-05     1.12224E-08    -2.36352E-03
.....................................
\end{verbatim} } \renewcommand{\baselinestretch}{1.07}\small\normalsize
\noindent
which is also followed by the associated angular distributions. This output 
block begins with:

{\small \begin{verbatim}

 8b2. Inelastic angular distributions    
   
    Level  1
   
 Angle       Total         Direct       Compound
  
   0.0    1.07320E-03    1.06392E-03    9.28143E-06
   2.0    1.07540E-03    1.06611E-03    9.28519E-06
   4.0    1.08196E-03    1.07266E-03    9.29647E-06
   6.0    1.09280E-03    1.08348E-03    9.31533E-06
   8.0    1.10774E-03    1.09840E-03    9.34185E-06
  10.0    1.12646E-03    1.11708E-03    9.37618E-06
  12.0    1.14844E-03    1.13902E-03    9.41848E-06
.....................................
\end{verbatim} } \renewcommand{\baselinestretch}{1.07}\small\normalsize
\noindent
Finally, the same is given for the $(n,p)$ and the other channels.
\subsubsection{Case g: Discrete gamma-ray production cross sections}
The gamma-ray intensity for each mother and daughter discrete level appearing 
in the reaction can be obtained with the following input file,

{\small \begin{verbatim}

#
# n-Nb093-14MeV-gamma
#
# General
#
projectile n
element nb
mass 93
energy 14.
#
# Output
#
outgamdis           y
\end{verbatim} } \renewcommand{\baselinestretch}{1.07}\small\normalsize
\noindent
For all discrete gamma-ray transitions, the intensity is printed.
For each nucleus, the initial level and the final level is given, the 
associated gamma energy and the cross section.
This output block begins with:  

{\small \begin{verbatim}

 10. Gamma-ray intensities

 Nuclide:  94Nb

     Initial level          Final level     Gamma Energy  Cross section

  no.  J/Pi    Ex         no.  J/Pi    Ex

   2   4.0+  0.0587  --->  1   3.0+  0.0409    0.01780    2.11853E-01
   3   6.0+  0.0787  --->  0   6.0+  0.0000    0.07867    2.12745E-01
   4   5.0+  0.1134  --->  0   6.0+  0.0000    0.11340    8.56884E-02
   4   5.0+  0.1134  --->  2   4.0+  0.0587    0.05470    3.36955E-02
   5   2.0-  0.1403  --->  1   3.0+  0.0409    0.09941    2.43882E-01
   6   2.0-  0.3016  --->  5   2.0-  0.1403    0.16126    5.42965E-02
   7   4.0+  0.3118  --->  2   4.0+  0.0587    0.25311    5.48060E-02
.....................................
\end{verbatim} } \renewcommand{\baselinestretch}{1.07}\small\normalsize
\noindent
When we discuss multiple incident energy runs in the other sample cases, 
we will see 
how the excitation functions for gamma production cross sections per level are
accumulated and how they can be written to separate files for easy processing.
\subsubsection{Case h: The full output file}    
In this sample case we print basically everything that can be printed in the
main output file for a single-energy reaction on a non-fissile nucleus. 
The input file is

{\small \begin{verbatim}

#
# n-Nb093-14MeV-full
#
# General
#
projectile n
element nb
mass 93
energy 14.
#
# Output
#
outbasic            y
outpreequilibrium   y
outspectra          y
outangle            y     
outlegendre         y     
ddxmode             2
outgamdis           y
partable            y    
\end{verbatim} } \renewcommand{\baselinestretch}{1.07}\small\normalsize
\noindent
resulting in an output file that contains all nuclear structure information, 
all partial results, and moreover all intermediate results of the 
calculation, as well as results of intermediate checking. Note that basically
all flags in the "Output" block on top of the output file are set to {\bf y},
the only exceptions being irrelevant for this sample case.
In addition to the 
output that is already described, various other output blocks are present.
First, since {\bf outbasic y} automatically means {\bf outomp y}, a block with 
optical model parameters is printed.
The optical model parameters for all
included particles are given as a function of incident energy.
This output block begins with:

{\small \begin{verbatim}

 ######### OPTICAL MODEL PARAMETERS ##########
 
           neutron  on  93Nb
 
  Energy    V     rv    av    W     rw    aw    Vd   rvd   avd    Wd   rwd   awd

   0.001  51.02 1.215 0.663  0.14 1.215 0.663  0.00 1.274 0.534  3.32 1.274 0.53
   0.002  51.02 1.215 0.663  0.14 1.215 0.663  0.00 1.274 0.534  3.32 1.274 0.53
   0.005  51.02 1.215 0.663  0.14 1.215 0.663  0.00 1.274 0.534  3.32 1.274 0.53
   0.010  51.02 1.215 0.663  0.14 1.215 0.663  0.00 1.274 0.534  3.33 1.274 0.53
   0.020  51.02 1.215 0.663  0.14 1.215 0.663  0.00 1.274 0.534  3.33 1.274 0.53
.....................................
\end{verbatim} } \renewcommand{\baselinestretch}{1.07}\small\normalsize
\noindent
In the next part, we print general quantities that are used throughout the 
nuclear reaction calculations, such as transmission coefficients and inverse 
reaction cross sections.
The transmission coefficients as
a function of energy are given for all particles included in the calculation.
Depending upon whether {\bf outtransenergy y} or {\bf outtransenergy n}, the 
transmission coefficient tables will be grouped per energy or per angular 
momentum, respectively. The latter option may be helpful to study the behavior 
of a particular transmission coefficient as a function of energy.
The default is {\bf outtransenergy n}, leading to the following output block,

{\small \begin{verbatim}

 ########## TRANSMISSION COEFFICIENTS AND INVERSE REACTION CROSS SECTIONS ######
 
 Transmission coefficients for incident neutron  at   0.00101 MeV
 
   L   T(L-1/2,L)   T(L+1/2,L)    Tav(L)  
  
   0  0.00000E+00  8.82145E-03  8.82145E-03
   1  1.45449E-04  2.56136E-04  2.19240E-04
  
 Transmission coefficients for incident neutron  at   0.00202 MeV
  
   L   T(L-1/2,L)   T(L+1/2,L)    Tav(L)
  
   0  0.00000E+00  1.24534E-02  1.24534E-02
   1  4.11628E-04  7.24866E-04  6.20454E-04
   2  2.55785E-08  1.88791E-08  2.15588E-08
  
 Transmission coefficients for incident neutron  at   0.00505 MeV
  
   L   T(L-1/2,L)   T(L+1/2,L)    Tav(L)
  
   0  0.00000E+00  1.96205E-02  1.96205E-02
   1  1.62890E-03  2.86779E-03  2.45483E-03
   2  2.52135E-07  1.86214E-07  2.12582E-07
.....................................
\end{verbatim} } \renewcommand{\baselinestretch}{1.07}\small\normalsize
\noindent
which is repeated for each included particle type.
Next, the (inverse) reaction cross sections 
is given for all particles on a LAB energy grid. For neutrons also the total 
elastic and total cross section on this energy grid is printed for completeness.
This output block begins with:

{\small \begin{verbatim}

 Total cross sections for neutron
 
     E        total      reaction    elastic   OMP reaction
 
   0.00101  1.1594E+04  6.2369E+03  5.3566E+03  6.2369E+03
   0.00202  1.0052E+04  4.7092E+03  5.3430E+03  4.7092E+03
   0.00505  8.8645E+03  3.5518E+03  5.3127E+03  3.5518E+03
   0.01011  8.4660E+03  3.1918E+03  5.2741E+03  3.1918E+03
   0.02022  8.4357E+03  3.2208E+03  5.2148E+03  3.2208E+03
   0.05054  8.9304E+03  3.8249E+03  5.1055E+03  3.8249E+03
   0.10109  9.6344E+03  4.5808E+03  5.0536E+03  4.5808E+03
   0.20217  1.0198E+04  5.0370E+03  5.1609E+03  5.0370E+03
   0.30326  1.0068E+04  4.7878E+03  5.2799E+03  4.7878E+03
   0.40434  9.6390E+03  4.3377E+03  5.3013E+03  4.3377E+03
   0.50543  9.1158E+03  3.8893E+03  5.2264E+03  3.8893E+03
   0.60651  8.5881E+03  3.5056E+03  5.0825E+03  3.5056E+03
   0.70760  8.0915E+03  3.1968E+03  4.8946E+03  3.1968E+03
.....................................
\end{verbatim} } \renewcommand{\baselinestretch}{1.07}\small\normalsize
\noindent
The final column "OMP reaction" gives the reaction cross section as obtained 
from the optical model. This is not necessary the same as the adopted reaction 
cross section of the middle column, since sometimes (especially for complex 
particles) this is overruled by 
systematics, see the {\bf sysreaction} keyword, (p.  \pageref{key:sysreaction}).
For the incident energy, we separately print the OMP parameters, the 
transmission coefficients and the shape elastic angular distribution,

{\small \begin{verbatim}

+++++++++ OPTICAL MODEL PARAMETERS FOR INCIDENT CHANNEL ++++++++++
 
           neutron  on  93Nb
 
  Energy    V     rv    av    W     rw    aw    Vd   rvd   avd    Wd   rwd   awd
   
  14.000  46.11 1.215 0.663  0.98 1.215 0.663  0.00 1.274 0.534  6.84 1.274 0.53
   
 Optical model results  
   
 Total cross section   : 3.9819E+03 mb
 Reaction cross section: 1.7706E+03 mb
 Elastic cross section : 2.2113E+03 mb
   
 Transmission coefficients for incident neutron  at  14.000 MeV
   
  L    T(L-1/2,L)   T(L+1/2,L)    Tav(L)
   
   0  0.00000E+00  7.46404E-01  7.46404E-01
   1  8.02491E-01  7.78050E-01  7.86197E-01
   2  7.77410E-01  8.08483E-01  7.96054E-01
   3  7.75550E-01  6.94249E-01  7.29092E-01
   4  9.15115E-01  9.53393E-01  9.36381E-01
   5  6.01374E-01  6.19435E-01  6.11226E-01
   6  7.00430E-01  4.72026E-01  5.77443E-01
   7  1.15102E-01  1.90743E-01  1.55444E-01
   8  1.59959E-02  1.88919E-02  1.75291E-02
   9  2.36267E-03  2.49532E-03  2.43249E-03
  10  3.55525E-04  3.61442E-04  3.58625E-04
  11  5.38245E-05  5.39431E-05  5.38864E-05
  12  8.20113E-06  8.17260E-06  8.18630E-06
  13  1.26174E-06  1.25428E-06  1.25788E-06

 Shape elastic scattering angular distribution

 Angle    Cross section

   0.0     6.69554E+03
   2.0     6.62134E+03
   4.0     6.40317E+03
   6.0     6.05390E+03
   8.0     5.59369E+03
  10.0     5.04824E+03
  12.0     4.44657E+03
  14.0     3.81867E+03
  16.0     3.19323E+03
  18.0     2.59561E+03
  20.0     2.04638E+03
.....................................
\end{verbatim} } \renewcommand{\baselinestretch}{1.07}\small\normalsize
\noindent
At some point during a run, TALYS has performed the direct reaction calculation
and the pre-equilibrium calculation. A table is printed which shows the part 
of the reaction population that is left for the formation of a compound 
nucleus. Since the 
pre-equilibrium cross sections are calculated on an emission energy grid, 
there is always a small numerical error when transferring these results to the 
excitation energy grid. The pre-equilibrium spectra are therefore normalized.
The output block looks as follows

{\small \begin{verbatim}

 ########## POPULATION CHECK ##########

 Particle Pre-equilibrium Population
  
 gamma        1.28644     1.27794
 neutron    428.71729   410.91696
 proton      25.24361    25.20203
 deuteron     3.50197     3.49032 
 triton       0.09480     0.09451
 helium-3     0.00000     0.00000
 alpha       26.72690    26.64193

 ++++++++++ Normalization of reaction cross section ++++++++++
   
 Reaction cross section          : 1770.63000 (A)
 Sum over T(j,l)                 : 1770.62537 (B)
 Compound nucleus formation c.s. : 1243.46240 (C)
 Ratio C/B                       :    0.70227
\end{verbatim} } \renewcommand{\baselinestretch}{1.07}\small\normalsize
\noindent
After the compound nucleus 
calculation, the results from the binary reaction are printed. First, the 
binary cross sections for the included outgoing particles are printed, 
followed by, if {\bf outspectra y}, the binary emission spectra. If also 
{\bf outcheck y}, the integral over the emission spectra is checked against the 
cross sections. The printed normalization factor has
been applied to the emission spectra. This output block begins with:

{\small \begin{verbatim}

 ########## BINARY CHANNELS ###########

 ++++++++++ BINARY CROSS SECTIONS ++++++++++

 gamma    channel to Z= 41 N= 53 ( 94Nb): 2.23665E+00
 neutron  channel to Z= 41 N= 52 ( 93Nb): 1.68950E+03
 proton   channel to Z= 40 N= 53 ( 93Zr): 3.79183E+01
 deuteron channel to Z= 40 N= 52 ( 92Zr): 1.04112E+01
 triton   channel to Z= 40 N= 51 ( 91Zr): 6.70826E-01
 helium-3 channel to Z= 39 N= 52 ( 91Y ): 2.35820E-08
 alpha    channel to Z= 39 N= 51 ( 90Y ): 2.98966E+01

 Binary emission spectra

  Energy   gamma       neutron     proton      deuteron    triton      helium-3 

   0.001 1.14462E-06 1.89788E+00 0.00000E+00 0.00000E+00 0.00000E+00 0.00000E+0
   0.002 2.28923E-06 3.74165E+00 0.00000E+00 0.00000E+00 0.00000E+00 0.00000E+0
   0.005 5.72311E-06 9.25456E+00 0.00000E+00 0.00000E+00 0.00000E+00 0.00000E+0
   0.010 1.14464E-05 1.84461E+01 0.00000E+00 0.00000E+00 0.00000E+00 0.00000E+0
   0.020 2.28940E-05 3.68901E+01 0.00000E+00 0.00000E+00 0.00000E+00 0.00000E+0
   0.050 5.72550E-05 9.26411E+01 0.00000E+00 0.00000E+00 0.00000E+00 0.00000E+0
   0.100 1.14633E-04 1.86293E+02 0.00000E+00 0.00000E+00 0.00000E+00 0.00000E+0
   0.200 2.30055E-04 3.28686E+02 2.64905E-30 1.35926E-43 0.00000E+00 0.00000E+0
   0.300 7.48114E-04 4.14446E+02 3.56019E-23 9.99027E-34 9.03852E-41 0.00000E+0
   0.400 1.81798E-03 4.97986E+02 5.12047E-19 1.48517E-27 3.14528E-33 0.00000E+0
   0.500 2.88939E-03 5.48371E+02 3.42302E-16 1.47275E-23 2.43692E-28 0.00000E+0
   0.600 3.96241E-03 5.39365E+02 4.07677E-14 1.28871E-20 9.22918E-25 0.00000E+0
   0.700 5.03710E-03 5.29973E+02 1.64259E-12 2.45195E-18 5.46266E-22 0.00000E+0
.....................................
 ++++++++++ CHECK OF INTEGRATED BINARY EMISSION SPECTRA ++++++++++
 
             Continuum cross section  Integrated spectrum  Compound normalizatio
    
 gamma              2.19555E+00           2.19555E+00           1.00011E+00     
 neutron            1.66025E+03           1.64343E+03           1.01805E+00     
 proton             3.43220E+01           3.43220E+01           1.00226E+00     
 deuteron           3.51689E+00           3.51689E+00           1.02829E+00     
 triton             9.54471E-02           9.54471E-02           1.01145E+00     
 helium-3           2.86109E-10           2.86109E-10           0.00000E+00     
 alpha              2.86177E+01           2.86176E+01           9.99189E-01     
\end{verbatim} } \renewcommand{\baselinestretch}{1.07}\small\normalsize
\noindent
Since {\bf outpopulation y}, the population that remains in the first set of 
residual nuclides after binary emission is printed,

{\small \begin{verbatim}

 ++++++++++ POPULATION AFTER BINARY EMISSION ++++++++++

 Population of Z= 41 N= 53 ( 94Nb) after binary gamma    emission: 2.23665E+00
Maximum excitation energy:  21.076 Discrete levels: 10 Continuum bins: 40 Conti
 
 bin    Ex    Popul.    J= 0.0-   J= 0.0+   J= 1.0-   J= 1.0+   J= 2.0-   J= 2.0

  0   0.000 3.206E-07 0.000E+00 0.000E+00 0.000E+00 0.000E+00 0.000E+00 0.000E+
  1   0.041 1.889E-07 0.000E+00 0.000E+00 0.000E+00 0.000E+00 0.000E+00 0.000E+
  2   0.059 2.525E-07 0.000E+00 0.000E+00 0.000E+00 0.000E+00 0.000E+00 0.000E+
  3   0.079 1.528E-03 0.000E+00 0.000E+00 0.000E+00 0.000E+00 0.000E+00 0.000E+
  4   0.113 3.443E-03 0.000E+00 0.000E+00 0.000E+00 0.000E+00 0.000E+00 0.000E+
  5   0.140 1.051E-02 0.000E+00 0.000E+00 0.000E+00 0.000E+00 1.051E-02 0.000E+
  6   0.302 9.579E-03 0.000E+00 0.000E+00 0.000E+00 0.000E+00 9.579E-03 0.000E+
  7   0.312 1.818E-03 0.000E+00 0.000E+00 0.000E+00 0.000E+00 0.000E+00 0.000E+
  8   0.334 4.717E-03 0.000E+00 0.000E+00 0.000E+00 0.000E+00 0.000E+00 0.000E+
  9   0.396 6.486E-03 0.000E+00 0.000E+00 0.000E+00 0.000E+00 0.000E+00 0.000E+
 10   0.450 3.014E-03 0.000E+00 0.000E+00 0.000E+00 0.000E+00 0.000E+00 0.000E+
 11   0.708 3.665E-02 2.036E-04 1.520E-04 1.592E-03 1.152E-03 3.293E-03 2.540E-
 12   1.224 5.702E-02 2.592E-04 1.935E-04 2.065E-03 1.494E-03 4.428E-03 3.415E-
.....................................
\end{verbatim} } \renewcommand{\baselinestretch}{1.07}\small\normalsize
\noindent
where in this case bins 0-10 concern discrete levels and bins 11-50 concern 
continuum bins.

After this output of the binary emission, we print
for each nuclide in the decay chain the population as a function of 
excitation energy, spin and parity {\em before} it decays. This loop starts 
with the initial compound nucleus and the nuclides formed by binary emission. 
When all excitation energy bins of the nucleus have been depleted, the final
production cross section (per ground state/isomer) is printed. The feeding from 
this nuclide to all its daughter nuclides is also given. If in addition 
{\bf outspectra y}, the emission spectra for all outgoing particles from this 
nucleus are printed. At high incident energies, when generally 
{\bf multipreeq y}, the result from multiple pre-equilibrium emission is 
printed (not included in this output). If {\bf outcheck y}, it is checked 
whether the integral over the 
emission spectra from this nucleus is equal to the corresponding feeding cross 
section. This output block begins with:

{\small \begin{verbatim}

 ########## MULTIPLE EMISSION ##########

 Population of Z= 41 N= 53 ( 94Nb) before decay: 3.58813E+00
Maximum excitation energy:  21.076 Discrete levels: 10 Continuum bins: 40 Conti

 bin    Ex    Popul.    J= 0.0    J= 1.0    J= 2.0    J= 3.0    J= 4.0    J= 5.0

  0   0.000 1.050E-06 0.000E+00 0.000E+00 0.000E+00 0.000E+00 0.000E+00 0.000E+
  1   0.041 6.181E-07 0.000E+00 0.000E+00 0.000E+00 6.181E-07 0.000E+00 0.000E+
  2   0.059 8.266E-07 0.000E+00 0.000E+00 0.000E+00 0.000E+00 8.266E-07 0.000E+
  3   0.079 1.525E-03 0.000E+00 0.000E+00 0.000E+00 0.000E+00 0.000E+00 0.000E+
  4   0.113 3.436E-03 0.000E+00 0.000E+00 0.000E+00 0.000E+00 0.000E+00 3.436E-
  5   0.140 1.049E-02 0.000E+00 0.000E+00 1.049E-02 0.000E+00 0.000E+00 0.000E+
  6   0.302 9.558E-03 0.000E+00 0.000E+00 9.558E-03 0.000E+00 0.000E+00 0.000E+
  7   0.312 1.814E-03 0.000E+00 0.000E+00 0.000E+00 0.000E+00 1.814E-03 0.000E+
  8   0.334 4.706E-03 0.000E+00 0.000E+00 0.000E+00 4.706E-03 0.000E+00 0.000E+
  9   0.396 6.472E-03 0.000E+00 0.000E+00 0.000E+00 6.472E-03 0.000E+00 0.000E+
 10   0.450 3.008E-03 0.000E+00 0.000E+00 0.000E+00 3.008E-03 0.000E+00 0.000E+
 11   0.708 3.657E-02 1.517E-04 1.150E-03 2.536E-03 3.531E-03 3.593E-03 2.702E-
 12   1.224 5.693E-02 1.939E-04 1.497E-03 3.419E-03 5.023E-03 5.489E-03 4.513E-
 13   1.739 6.580E-02 1.843E-04 1.443E-03 3.395E-03 5.210E-03 6.036E-03 5.335E-
 14   2.255 7.347E-02 1.744E-04 1.381E-03 3.323E-03 5.274E-03 6.390E-03 5.973E-
.....................................
 Emitted flux per excitation energy bin of Z= 41 N= 53 ( 94Nb):
   
bin    Ex      gamma       neutron     proton      deuteron    triton      heli

  0   0.000 0.00000E+00 0.00000E+00 0.00000E+00 0.00000E+00 0.00000E+00 0.00000
  1   0.041 0.00000E+00 0.00000E+00 0.00000E+00 0.00000E+00 0.00000E+00 0.00000
  2   0.059 2.11853E-01 0.00000E+00 0.00000E+00 0.00000E+00 0.00000E+00 0.00000
  3   0.079 2.12745E-01 0.00000E+00 0.00000E+00 0.00000E+00 0.00000E+00 0.00000
  4   0.113 1.19384E-01 0.00000E+00 0.00000E+00 0.00000E+00 0.00000E+00 0.00000
  5   0.140 2.43882E-01 0.00000E+00 0.00000E+00 0.00000E+00 0.00000E+00 0.00000
  6   0.302 5.42965E-02 0.00000E+00 0.00000E+00 0.00000E+00 0.00000E+00 0.00000
.....................................
 Emission cross sections to residual nuclei from Z= 41 N= 53 ( 94Nb):
   
 gamma    channel to Z= 41 N= 53 ( 94Nb): 2.66180E+00
 neutron  channel to Z= 41 N= 52 ( 93Nb): 1.05107E+00
 proton   channel to Z= 40 N= 53 ( 93Zr): 1.67352E-03
 deuteron channel to Z= 40 N= 52 ( 92Zr): 1.07608E-06
 triton   channel to Z= 40 N= 51 ( 91Zr): 2.06095E-08
 helium-3 channel to Z= 39 N= 52 ( 91Y ): 1.07027E-16
 alpha    channel to Z= 39 N= 51 ( 90Y ): 4.69800E-04
   
 Emission spectra from Z= 41 N= 53 ( 94Nb):
   
  Energy   gamma       neutron     proton      deuteron    triton      helium-3 

   0.001 1.14218E-02 5.11503E-03 0.00000E+00 0.00000E+00 0.00000E+00 0.00000E+0
   0.002 1.19276E-02 1.10501E-02 0.00000E+00 0.00000E+00 0.00000E+00 0.00000E+0
   0.005 1.34434E-02 3.25383E-02 0.00000E+00 0.00000E+00 0.00000E+00 0.00000E+0
   0.010 1.59698E-02 7.24291E-02 0.00000E+00 0.00000E+00 0.00000E+00 0.00000E+0
   0.020 2.10227E-02 1.27402E-01 0.00000E+00 0.00000E+00 0.00000E+00 0.00000E+0
   0.050 4.45166E-02 2.33433E-01 0.00000E+00 0.00000E+00 0.00000E+00 0.00000E+0
.....................................
 ++++++++++ CHECK OF INTEGRATED EMISSION SPECTRA ++++++++++
   
             Cross section   Integrated spectrum  Average emission energy
 
 gamma        2.66180E+00     2.66180E+00             1.776
 neutron      1.05107E+00     1.05107E+00             1.174
 proton       1.67352E-03     1.67334E-03             5.377
 deuteron     1.07608E-06     7.97929E-07             5.635
 triton       2.06095E-08     0.00000E+00             0.000
 helium-3     1.07027E-16     0.00000E+00             0.000
 alpha        4.69800E-04     4.69563E-04            10.831
   
 Final production cross section of Z= 41 N= 53 ( 94Nb):    
  
 Total       : 1.18344E+00
 Ground state: 5.75062E-01
 Level  1    : 6.08379E-01
\end{verbatim} } \renewcommand{\baselinestretch}{1.07}\small\normalsize
\noindent
Note that once a new nucleus is encountered in the reaction chain, all nuclear 
structure information for that nucleus is printed as well.
\subsubsection{Case i: No output at all}
It is even possible to have an empty output file. With the following input file,

{\small \begin{verbatim}

#
# General
#
projectile n
element nb
mass 93
energy 14.
#
# Output
#
outmain n
\end{verbatim} } \renewcommand{\baselinestretch}{1.07}\small\normalsize
\noindent
it is specified that even the main output should be suppressed. 
The sample output file should be empty. This can be helpful 
when TALYS is invoked as a subroutine from other programs and the output from 
TALYS is not required (if the communication is done e.g. through 
shared arrays or subroutine variables). We have not yet used this option
ourselves.
\end{samplecase}
